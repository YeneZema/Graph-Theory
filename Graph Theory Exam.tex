\documentclass[paper=a4, fontsize=12pt]{scrartcl} % A4 paper and 11pt font size

\usepackage[T1]{fontenc} % Use 8-bit encoding that has 256 glyphs
\usepackage{fourier} % Use the Adobe Utopia font for the document - comment this line to return to the LaTeX default
\usepackage[english]{babel} % English language/hyphenation
\usepackage{amsmath,amsfonts,amsthm,amssymb} % Math packages
\usepackage{graphicx,wrapfig,lipsum}
\usepackage{fancyhdr} % Custom headers and footers
\pagestyle{fancyplain} % Makes all pages in the document conform to the custom headers and footers
\usepackage[margin=1in]{geometry}  % set the margins to 1in on all sides
\usepackage[hidelinks]{hyperref}
\usepackage{cleveref}

\newtheorem{thm}{Theorem}[section]
\newtheorem{lem}[thm]{Lemma}
\newtheorem{prop}[thm]{Proposition}
\newtheorem{cor}[thm]{Corollary}
\newtheorem{conj}[thm]{Conjecture}
\usepackage{sectsty} % Allows customizing section commands
\allsectionsfont{\centering \normalfont\scshape} % Make all sections centered, the default font and small caps



\fancyhead{} % No page header - if you want one, create it in the same way as the footers below
\fancyfoot[L]{} % Empty left footer
\fancyfoot[C]{} % Empty center footer
\fancyfoot[R]{\thepage} % Page numbering for right footer
\renewcommand{\headrulewidth}{0pt} % Remove header underlines
\renewcommand{\footrulewidth}{0pt} % Remove footer underlines
\setlength{\headheight}{13.6pt} % Customize the height of the header

% various theorems, numbered by section

\theoremstyle{definition}
\newtheorem{defn}[thm]{Definition}
\newtheorem{defns}[thm]{Definitions}
\newtheorem{con}[thm]{Construction}
\newtheorem{exmp}[thm]{Example}
\newtheorem{exmps}[thm]{Examples}
\newtheorem{notn}[thm]{Notation}
\newtheorem{notns}[thm]{Notations}
\newtheorem{addm}[thm]{Addendum}
\newtheorem{exer}[thm]{Exercise}

\theoremstyle{remark}
\newtheorem{rem}[thm]{Remark}
\newtheorem{rems}[thm]{Remarks}
\newtheorem{warn}[thm]{Warning}
\newtheorem{sch}[thm]{Scholium}
\DeclareMathOperator{\id}{id}

\newcommand{\bd}[1]{\mathbf{#1}}  % for bolding symbols
\newcommand{\RR}{\mathbb{R}}      % for Real numbers
\newcommand{\ZZ}{\mathbb{Z}}      % for Integers
\newcommand{\col}[1]{\left[\begin{matrix} #1 \end{matrix} \right]}
\newcommand{\comb}[2]{\binom{#1^2 + #2^2}{#1+#2}}
%----------------------------------------------------------------------------------------
%	TITLE SECTION
%----------------------------------------------------------------------------------------

\newcommand{\horrule}[1]{\rule{\linewidth}{#1}} % Create horizontal rule command with 1 argument of height

\title{
		%\vspace{-1in} 	
		\usefont{OT1}{bch}{b}{n}
		\normalfont \normalsize \textsc{Exam 1} \\ [25pt]
		\horrule{0.5pt} \\[0.4cm]
		\huge Graph Theory \\
		\horrule{2pt} \\[0.5cm]
}
\author{
		\normalfont 								\normalsize
        Miliyon Tilahun (NSR/4137/05)\\[-3pt]		\normalsize
        \today
}
\date{}


%%% Begin document
\begin{document}
\maketitle

\subsection*{Solve the following problems by showing all the necessary steps.}

\begin{enumerate}
  \item Show that either a graph or its complement is connected.
  \item Show that at least two vertices in a graph of order $2$  or more have the same degree.
  \item Show that any non trivial tree has at least two pendant(degree $1$) vertices.
\end{enumerate}

\subsection*{Solution}

\begin{enumerate}
  \item Let $G$ be a graph(finite)\\
  Suppose $G$ is disconnected and assume $G_1,G_2,\ldots,G_n$ the connected components of $G$.\\
  Now, $G = G_1+G_2+\ldots+G_n$ where $+$ is the disjoint union operator.\\
  Thus,
  \begin{align*}
   \overline{G} &=\overline{G_1+G_2+\ldots+G_n}\\
              &=\overline{G_1}\vee\overline{G_2}\vee \ldots \vee\overline{G_n}
  \end{align*}
  where $\vee$ is the join operator.\\
  Hence $\overline{G}$ is connected. i.e. whenever $G$ is disconnected $\overline{G}$ is connected.\\
  That was to be shown!

  \item Let $G$ be a graph of order $n\geq2$.\\
  The possible values for $\deg(v_i)$ are $n-1,n-2,\ldots,2,1$\\
  It is a kind of pigeonhole principle; we have $n$ vertices(pigeons) and $n-1$ possible values of degree(hole).\\
  Which implies there has to be at least two vertices(pigeons) with the same degree(for pigeons in the same hole).
  \item Suppose not(there is at most one vertex of degree one(pendant)) and assume $T$ is order $n\geq2$.\\
  We know that $||T||=n-1$ (which can be proved using induction).\\
  Thus, handshaking lemma tells us
  \begin{align}\label{pendant1}
  \sum_{v\in V(T)}\deg(v)=2(n-1)
  \end{align}
  But from our assumption there is at most one pendant vertex. This implies
  \begin{align}\label{pendant2}
  \sum_{v\in V(T)}\deg(v)\geq 2(n-1)=2n-2+1=2n-1
  \end{align}
  Putting (\ref{pendant1}) and (\ref{pendant2}) together
  \begin{align*}
  2(n-1)\geq 2n-1 \qquad \text{ which is a contradiction!}
  \end{align*}
  Therefore, our assumption was false hence.
\end{enumerate}
\end{document}
